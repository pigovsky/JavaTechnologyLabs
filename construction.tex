\section{Роль дисципліни ``Конструювання ПЗ'' при виконанні лабораторних робіт з ``Технології Java'' та ``Технології .NET''}
Згідно з [IEEE. SWEBOK Guide V3. CHAPTER 4. SOFTWARE CONSTRUCTION] конструювання ПЗ покликане оптимізувати процес розробки ПЗ з метою:
\begin{itemize}
\item мінімізації складності,
\item готовності до змін,
\item верифікації,
\item повторного використання,
\item стандартизації
\end{itemize}

Підчас лабораторних робіт з ``Конструювання ПЗ'' студенти, застосовуючи теоретичні рекомендації щодо якісного конструювання досягатимуть вищезгаданих цілей в текстах лабораторних робіт технологічно-орієнтованих дисциплін ``Технологія Java'' та ``Технологія .NET''.

Таким чином лабораторні роботи з дисципліни ``Конструювання ПЗ'' виконуватимуться в рамках текстів програмного коду лабораторних робіт з дисциплін ``Технологія Java'' та ``Технологія .NET''.

На лабораторних заняттях з ``Конструювання ПЗ'' кожен файл програмного коду лабораторних робіт з дисциплін ``Технологія Java'' та ``Технологія .NET'' буде оптимізований відповідно до наступних заходів:
\begin{itemize} 
\item конвенції щодо оформлення коду на різних мовах програмування (відступи, дужки, іменування): для Java --- [\url{http://www.oracle.com/technetwork/java/codeconvtoc-136057.html}], 
для C\#.NET --- [\url{http://msdn.microsoft.com/en-us/library/ff926074.aspx}],
\item користування засобами автоматизованого документування коду на основі анотацій: javadoc [\url{http://www.oracle.com/technetwork/java/javase/documentation/index-137868.html}] для Java; XML Documentation [\url{https://www.simple-talk.com/dotnet/.net-tools/taming-sandcastle-a-.net-programmers-guide-to-documenting-your-code/}] та NDoc для .NET, 
\item теоретичні засоби мінімізації складності за допомогою грамотної інкапсуляції та дизайн паттернів (на основі фундаментальної книги Стівена Макконнелла про конструювання ПЗ),
\item модулі повторного використання (reusable units): бібліотеки класів, фреймворки, засоби розробки (SDK),
\item засоби інспекції коду (виявлення можливих ділянок оптимізації, усунення застарілих чи громістких конструкцій) і рефекторингу в інтегрованих середовищах розробки (IDE): автоматизованої інкапсуляції полів, автоматичного створення методів і змінних (extract method, variable), винесення інтерфейсу та супер(базового)класу, створення конструкторів, білдерів і фабрик об'єктів,
\item засоби модульного (unit) тестування: JUnit [\url{http://junit.org/}] для Java; NUnit чи MSUnit [\url{http://msdn.microsoft.com/en-us/library/hh598960.aspx}] для .NET,
\item засоби автоматизації функціонального (інтеграційного) тестування, або, іншими словами, UI тестування, наприклад, сервіс \url{http://testfairy.com} та SDK засіб uiautomator для Java-Android [\url{http://developer.android.com/tools/help/uiautomator/index.html}].
\end{itemize}