\section{Лабораторна робота 1. Типи даних. Інкапсуляція. Агрегація}
 

Аудиторний час на виконання лабораторної роботи --- 16 годин.

Мета --- набуття практичних навичок:
\begin{enumerate}
\item Створення виконуваних класів, їхньої компіляції та виконання в середовищі JDK.
\item Створення класів, що інкапсулюють поля різних типів.
\item Реалізація принципу агрегації засобами масивів.
\end{enumerate}

\subsection{Анотація}
Консольний додаток, що створює масив об'єктів-працівників, що містять випадкові дані (ім’я, дата народження, стаж), друкує її і забуває. 
Класи: App, Worker, WorkerBuilder.

\subsection{Теоретичні відомості}
Основою будь-якої прикладної програми у Java є виконуваний клас. Виконуваним вважається клас, який є загальнодоступним (public), тобто видимим за межами власного пакету (докладніше про пакети і видимість --- згодом), а також містить статичну функцію
\begin{lstlisting}
public static void main(String[]a){/* code */}
\end{lstlisting}
або
\begin{lstlisting}
public static void main(String a[]){/* code */}
\end{lstlisting}
або
\begin{lstlisting}
public static void main(String ...a){/* code */}
\end{lstlisting}

У лістингу 1 наведено приклад простого виконуваного класу.

{\bf Лістинг 1.
Приклад виконуваного класу }
\begin{lstlisting}
public class Welcome{
    public static void main(String...a){
        System.out.println("Welcome, dear "+
            "third-form students of PZAS!");
    }
}
\end{lstlisting}
Первинний код класу можна підготувати у будь-якому текстовому редакторі, навіть у notepad, хоча суттєво зручніше працювати у редакторі, що виділяє мовні конструкції кольором, наприклад, Notepad++ або у інтегрованому середовищі розробки, наприклад, Eclipse чи IntelliJ IDEA. Проте, майте на увазі, що це повинен бути справді текстовий редактор (англійською plain text editor), а не текстові процесори на зразок MS Word, які добавляють до текстів спеціальні символи форматування.
Імена файла та виконуваного класу, що в ньому знаходиться повинні співпадати. Файл мусить мати розширення ``.java''. Наприклад, первинний код виконуваного класу з лістингу 1 повинен бути збереженим будь-де у файлі з назвою Welcome.java. 
Компіляція первинного коду у байт-код, придатний до виконання віртуальною машиною Java (JVM) можна здійснити у середовищі розробки або з командного рядка.

Якщо Ви вирішили працювати у середовищі Ecilpse чи IDEA, то запустіть його. Ми рекомендуємо користуватися IntelliJ IDEA Community Edition для розробки під JavaSE та Android Studio для розробки під Android, оскільки вони мають хороші засоби інспекції коду, рефакторингу, використовують сучасну технологію збирання gradle з репозиторіями модулів повторного використання maven. 

Створіть новий проект, клацнувши на кнопку Create New Project.
Вкажіть ``Workers catalogue by <Ваше прізвище>'' як назву проекту. Звісно вставте Ваше справжнє прізвище замість <Ваше прізвище>. 
У полі Project location вкажіть шлях до каталогу, де будуть зберігатися файли проекту. Будь ласка, встановіть цей шлях на носії, що придатний для довготривалого збереження даних, бо ці файли будуть використовуватися для оцінювання Вашої роботи.
У випадаючому списку Project SDK оберіть 1.7 (java version "1.7.x\_xx") або 1.8 (java version "1.8.x\_xx"). Клацніть next, finish і в ``Project explorer'' Ви побачите, що новий проект створено.
 
Розширте список його файлів, клацнувши трикутничок ``expose''. 
 Потім клацніть правою клавішею на ``src'', оберіть New->Package і створіть пакет ``com.<Ваше прізвище>.lab1''. В схожий спосіб, клацнувши правою клавішею на пакеті, який щойно утворився, створіть підпакети ``model'', ``util'' та ``application''.
 
За допомогою контекстного меню New->Java Class створіть клас App в пакеті application. Після натиснення ``OK'', появиться вікно редагування коду класу. Модифікуйте його відповідно до лістингу 1.
Клацніть на кнопку ``Make'', що схематично показана зеленою стрілочкою вниз та двійковими числами. Після закінчення процесу збирання, оберіть праворуч із випадаючого списку  конфігурацію ``App'' і клацніть зелену кнопку-трикутничок ``Run''.
 
Після запуску програми побачите результат її виконання у віконці Console.

Якщо у Вас, з якихось причин, немає доступу до інтегрованого середовища розробки, наберіть лістинг 1 у будь-якому текстовому редакторі, збережіть його у папці com/<Ваше прізвище>/lab1/application і виконайте компіляцію з командного рядка: 

javac App.java

Примітка: якщо Ви працюєте в ОС, що ігнорує регістр букв у назвах файлів, то компіляцію можна виконати й командою javac app.java або, навіть JaVaC aPp.jAvA

Якщо результатом спроби компіляції є повідомлення про те, що javac не є ні внутрішньою, ні зовнішньою командою, ні виконуваною програмою, ні пакетним файлом, то або Ви не встановили JDK або не налаштовано змінну середовища з шляхом до його виконуваних файлів. Дізнатись про те, де отримати JDK і як налаштувати змінну середовища PATH можна з додатку А.

Після виконання компіляції у поточній директорії появиться файл App.class, що містить байт-код виконуваного класу. Щоб запустити клас на виконання слід у командному рядку написати 

java -cp <шлях/до/папки/com> com.<Ваше прізвище>.lab1.application.App

або, коли <шлях/до/папки/com> записано в змінній середовища CLASSPATH, то просто

java com.<Ваше прізвище>.lab1.application.App

Примітка: на відміну від команди компілювання, команда запуску виконуваного класу чутлива до регістру букв у його назві. Це справедливо для всіх платформ, навіть MS Windows.

\subsection{Хід роботи}
Напишіть програму на мові Java в якій будуть наступні модулі (файли *.java з текстом класів):
\begin{enumerate}
\item клас Worker в пакеті com.<Ваше прізвище>.lab1.model з полями:
	\begin{itemize}
	\item прізвище (String),
	\item імена (String),
	\item рік прийому на роботу (Integer),
	\item рік народження (Integer),
	\item основне місце роботи (String)
	\end{itemize}
   в класі повинен бути переозначений метод toString() в такий спосіб, щоб він повертав інформацію про працівника у вигляді читабельної стрічки; всі поля повинні бути інкапсульовані і доступні за допомогою методів getters-setters; клас повинен містити п'ять конструкторів, кожен з яких ініціалізує від одного до п'яти полів,
\item  клас com.<Ваше прізвище>.lab1.util.WorkerBuilder, що реалізує зразок проектування (дизайн паттерн) builder в такий спосіб, що можна створювати працівників за допомогою конструкції на зразок:
\begin{lstlisting}
Worker worker = new WorkerBuilder().surname("Pigovsky")
    .names("Yuriy Romanovych")
        .yearOfBirth(1983)
            .yearOfEmployment(2004)
                .build();
\end{lstlisting}
\item статичний метод WorkerBuilder.generateWorkers(), що, користуючись вищезгаданим WorkerBuilder, повертає масив 10 працівників з випадковими іменами і датою народження 
\item виконуваний клас com.<Ваше прізвище>.lab1.application.App, що виводить в консоль всіх працівників, згенерованих методом WorkerBuilder.generateWorkers()
\end{enumerate}

 При реалізації методів доступу (getters, setters) до  полів класу Worker користуйтеся  рефакторінгом. Клацніть правою кнопкою миші на тілі класу і у контекстному меню оберіть Refactor->Encapsulate Fields...
Відзначне галочками потрібні поля, клацніть ОК і до тексту класу добавляться методи доступу на зразок
\begin{lstlisting}
    public void setSurname(String surname) {
        this.surname = surname;
    }
    public String getSurname() {
        return surname;
    }
\end{lstlisting}



Для того, щоб створити конструктор, що ініціалізує будь-які чи всі поля класу натисніть, знаходячись у файлі з класом, комбінацію клавіш <Alt>+<Insert> (або оберіть з контекстного меню опцію Generate) і клацніть ``Constructor''. У вікні, що появиться, виділіть необхідні поля і натисніть Refactor.

Щоб створити WorkerBuilder оберіть з контекстного меню Refactor->Replace constructor with builder...


Для автоматичного форматування тексту програми  натисніть комбінацію клавіш <Ctrl>+<L>. 

Вивід в консоль виконуйте методом System.out.println().

Переозначення методу toString можна виконати написавши частину його імені в тексті класу і натиснувши комбінацію кнопок автодоповнення <Ctrl>+<ПРОБІЛ>. Цим же способом можна користовуватися для автодоповнення імен, визначення списку аргументів метода і т.д. Комбінація кнопок <Ctrl>+<Q> показує швидку підказку до метода на якому стоїть курсор, а <Ctrl>+<P> показує набір його аргументів.

\subsection{Порядок виконання програми}
Програма повинна запускатися з класу ``App'', тобто він виступає виконуваним класом. Після запуску програми функція main повинна у циклі друкувати всіх працівників, що були згенеровані у масиві. 

\subsection{Зміст звіту}

Звітом з виконання лабораторної роботи є файли проекту.
 
