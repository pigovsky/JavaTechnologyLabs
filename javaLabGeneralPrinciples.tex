\section{Загальні принципи звітування і оцінювання лабораторних робіт з ``Технології Java''}

Роль лабораторних робіт НЕ полягає в тому, щоб дати студентам об’ємне завдання, на виконання якого вимагається багато як аудиторного так і позааудиторного часу, а в тому, щоб студент набув практичні знання і навички, які може продемонструвати безпосередньо в присутності викладача. 

Звітом з виконання всіх лабораторних робіт з ``Технології Java'' виступає оприлюднений (public) репозиторій на одному з загальновідомих сервісів хостингу репозиторіїв, наприклад, hithub.com, gitlab.com чи bitbucket.com. Жодні паперові звіти в ніяких формах не приймаються. 

Оприлюднений у репозиторії звіт є необхідною умовою захисту лабораторних робіт, проте, не означає, що студент обов’язково отримає позитивну оцінку, чи отримає її взагалі.

Оцінка виставляється виключно на основі якості виконання студентом коротких завдань з програмування, що пов’язані з темою лабораторної роботи, у безпосередній присутності викладача.
