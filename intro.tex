\section*{Вступ}
Java --- це мова програмування та комп'ютерна платформа, вперше  реалізована компанією Sun Microsystems в 1995. Вже існує велика кількість застосунків та вебсайтів, що спираються на технологію Java, але щодня появляються все нові й нові. Java є швидкодіючою, безпечною та надійною технологією [\url{http://www.java.com/en/download/faq/whatis_java.xml}]. 

Технологія Java [\url{http://www.oracle.com/technetwork/topics/newtojava/java-technology-concept-map-150250.pdf}] використовується на дуже багатьох платформах (під платформою розуміється комбінація обладнання та операційної системи):
від лептопів до дейтацентрів (datacenter), ігрових консолей до наукових суперкомп'ютерів, мобільних телефонів до хмарних технологій Internet.

Дисципліна ``Технологія Java'' описує основні питання платформи Java, а саме:
\begin{itemize}
\item особливості Java як мови програмування [\url{http://docs.oracle.com/javase/specs/jls/se8/html/index.html}]: структура проекту, пакети як аналог простору імен в С++ та С\#, інтерфейс Iterable, узагальнені класи, успадкування, область видимості, анонімна реалізація інтерфейсів як аналог lambda операторів, з березня 2014 року появився Java SE 8 з lambda виразами, посиланнями на метод (аналог делегатів) і функціональними інтерфейсами,  порівняння об'єктів та значень, переозначення методів equals та clone, обчислювальні потоки чи нитки, засоби синхронізації на основі моніторів, ввід-вивід, користування командним рядком java, javac та змінними середовища CLASSPATH, JAVA\_HOME, ANDROID\_HOME,
\item засоби автоматизації збирання, тестування і публікування Ant та gradle, репозиторії бібліотек maven,
\item інтегроване середовище розробки InelliJ Idea чи Android Studio (засоби інспекції коду та рефакторингу),
\item основи розробки для J2SE (файловий ввід-вивід, серіалізація),
\item основи розробки для ОС Android: життєвий цикл Activity, робота з ресурсами (стрічки, зображення, звук), форматування layout, обробка подій, робота зі списками ListView, перенесення даних з однієї активності в іншу, особливості проектування застосунків для Андроід (синглтон застосунку).
\end{itemize}

