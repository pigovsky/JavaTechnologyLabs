\section{Лабораторна робота 2. Наслідування. Поліморфізм }
 

Аудиторний час на виконання лабораторної роботи – 10 годин.

\subsection{Анотація}
Розвиток попередньої програми, але з циклом наповнення колекції працівників, де користувач повинен мати можливість вибрати два види працівників: Науковець або Верстатник. Ці працівники описуються однойменними класами, нащадками класу Працівник.
Науковець має поля для збереження стрічок наукового ступеня та вченого звання.
Верстатник має поле для збереження розряду та стрічки з описом верстатів, на яких може працювати.
На кожній ітерації циклу введення інформації про працівника програма запитуватиме його тип (1 – для науковця, 2 – для верстатника, 0 – для завершення вводу колекції працівників).

Хід роботи
Модифікувати програму лабораторної роботи №1 таким чином, щоб вона реалізувала наступну діаграму класів UML. 
 

Кожне приватне поле повинне супроводжуватися getter-setter, незважаючи на те, що це не показано на діаграмі явно. 
Метод ввести() в класі „Працівник” призначений для заповнення всіх полів класу з клавіатури за допомогою класу java.util.Scanner. Методи ввести(), переозначені (@Override) в класах „Науковець” та „Верстатник” викликають одноіменний метод суперкласу та запитують з клавіатури специфічні для себе поля (наукову ступінь, вчене звання або розряд та список верстатів).
Метод ввестиПрацівника() у класі „ВсіПрацівники” повинен забезпечувати вибір типу працівника і, відповідно, викликати метод ввести(), що переозначений для науковця та верстатника. Вибір типу працівника можна запропонувати у такій формі: „1 – для науковця, 2 – для верстатника, 0 – для завершення вводу колекції працівників”.
У класі „Головний” символ $, згідно прийнятої домовленості, поряд з методом main означає, що це статичний метод. Цикл введення працівників повинен знаходитися в методі main. 
Метод ВсіПрацівники.надрукувати() повинен викликати метод надрукувати() всіх об’єктів, що є елементами колекції працівників. При цьому методи Науковець.надрукувати() та Верстатник.надрукувати() друкують різну інформацію, відповідно до типу працівника, проте користуються методом надрукувати() свого суперкласу „Працівник” для друку спільних для всіх працівників властивостей. Метод надрукувати()  у класі Працівник є віртуальним, бо переозначений (@Override) в нащадках Науковець і Верстатник. Віртуальні (і абстрактні) методи на UML позначаються нахиленим шрифтом: надрукувати().
Створення класів Верстатник та Науковець відбувається шляхом наслідування класу Працівник. Коли клас Працівник має лише один конструктор, що вимагає задання аргументів, то після створення нового класу на його основі буде відображено наступну помилку
 

Усунути цю помилку можна добавивши конструктор по-замовчуванню до класу Працівник або натиснувши комбінацію клавіш <Ctrl>+<1>, знаходячись на стрічці з помилкою і обравши з меню, що появиться перший пункт:
 

Після цього код набуде такого вигляду:
import java.util.Date;


public class Верстатник extends Працівник {

	public Верстатник(String основнеМісцеРоботи, String імена, String прізвище,
			int кількістьПовнихРоківСтажу, Date датаНародження) {
		super(основнеМісцеРоботи, імена, прізвище, кількістьПовнихРоківСтажу,
				датаНародження);
		// TODO Auto-generated constructor stub
	}

}

Для того, щоб переозначити метод надрукувати() у класах Науковець та Верстатник можна скористатися контекстним меню Source (Alt+Shift+S) -> Override/Implement Methods…, як показано нижче
 

у вікні, що появиться слід обрати метод надрукувати(), як показано на рисунку нижче

 

В результаті появиться наступне оголошення методу

	@Override
	public void надрукувати() {
		// TODO Auto-generated method stub
		super.надрукувати();
	}

Тіло методу починається із виклику методу надрукувати(), що оголошений у батьківському класі „Працівник”. Після цього методу слід добавити код для виведення специфічних для Науковця та верстатника полів.

Приклад діалогу з програмою 
1 – для науковця, 2 – для верстатника, 0 – для завершення вводу колекції працівників
1
Введіть прізвище працівника Піговський
Введіть імена працівника Юрій Романович
Введіть дату народження працівника 11.02.1983
Введіть стаж працівника 7
Введіть основне місце роботи працівника ТНЕУ
Наукова ступінь к.т.н.
Вчене звання -
1 – для науковця, 2 – для верстатника, 0 – для завершення вводу колекції працівників
2
Введіть прізвище працівника Висоцький
Введіть імена працівника Василь Васильович
Введіть дату народження працівника 01.11.1990
Введіть стаж працівника 4
Введіть основне місце роботи працівника Оріон
Розряд 3
Введіть назву верстата (0 для виходу) Токарний
Введіть назву верстата (0 для виходу) Фрезерувальний
Введіть назву верстата (0 для виходу) з ЧПУ
Введіть назву верстата (0 для виходу) 0
1 – для науковця, 2 – для верстатника, 0 – для завершення вводу колекції працівників
0
Піговський Юрій Романович, Fri Feb 11 00:00:00 EET 1983 р.н., працює в ТНЕУ має стаж 7 років
	к.т.н., -
Висоцький Василь Васильович, Thu Nov 01 00:00:00 EET 1990 р.н., працює в Оріон має стаж 4 років
	має розряд 3
	та працює на верстатах: Токарний, Фрезерувальний, з ЧПУ

Зміст звіту

Звітом з виконання лабораторної роботи є файли проекту.
 
